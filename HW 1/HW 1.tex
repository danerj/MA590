\documentclass[11pt]{article}

\usepackage{amssymb}
\usepackage{amsmath}
\usepackage{amsthm}
\usepackage{indentfirst}

\title{Imaging of discontinuities in the inverse
scattering problem by inversion of a causal generalized Radon transform, G. Beylkin}
\date{}

\begin{document}
\maketitle

\noindent Journal of Mathematical Physics 26, 99 (1985); https://doi.org/10.1063/1.526755
Submitted: 08 May 1984 . Accepted: 20 July 1984 . Published Online: 04 June 1998\\

\null\hfill\begin{tabular}[t]{l@{}}
  \text{Dane Johnson} \\
  \text{MA 590 Homework 1}
\end{tabular}\\

This paper discusses {\bf the solution of the linear inverse scattering problem}. The inverse scattering problem arises in wave propagation in fluids with constant and variable densities and in elastic solids. One may need to interpret collected data using what the author calls nondestructive evaluation. They give the examples of seismic reflection data and ultrasound reflectivity data as instances where data might be collected but directly investigating the source of the data would be either impossible or undesirable. {\bf The inverse scattering problem is nonlinear and in the paper linearization is accomplished using a perturbation technique}. This linearization process leads to the integral equation (this appears to be a {\bf continuous model}):

$$\nu (k,\xi,\eta) = (-ik)^{n-1} \int_{X} f(x) a(x,\xi,\eta)e^{ik\phi(x,\xi,\eta)}\,,$$

where $X$ is the domain of definition of the {\bf unknown function} $f(x)$, $\xi$ and $\eta$ are points on the boundary $\partial X$ corresponding to receiver and source locations, and $k$ is the wave number. The phase function $\phi(x,\xi,\eta)$ can be broken down as the sum of two other phase functions that each satisfy two variable pde's. {\bf The integral equation is solved using asymptotic methods and migration schemes}\\

The goal is to characterize the function $f$ using observations of the scattered field on the boundary $\partial X$ of the specified region $X$, as generated by a known incident field. That is, established scientific methods are used to collect information about the process we want to study along the periphery of whatever object houses this process in order to infer the underlying mechanism that would have produced data that can be measured feasibly. {\bf The physical laws that govern $f$ must necessarily be related to wave propagation. Whether or not $f$ is linear is never mentioned in the paper, however}. The exact makeup of the available data would really depend on the applying the more general results of this paper to one particular application, but would be some measurement that arises through the propagation of a wave toward the surface of an object. {\bf The author solve this problem using discrete data points} (at least in the parts that I could vaguely understand), {\bf but does not discuss the possible sources of noise}. \\

To find an article I first looked through many papers in the journal \textit{Inverse Problems}. I was unable to find papers that ever explicitly mentioned what exactly the inverse problem they were attempting to solve was, the sources of data, the nature of the forward operator, etc. Since most of the papers were too complicated to actually determine the answers to the questions of the assignment I thought I might have more luck searching for an older article on inverse problems. It occurred to me that the wikipedia page on inverse problems might discuss seminal results in this area of math and that I might find more clearly written articles mentioned here. I was correct and gained access to this paper through the WPI library. My main interest in this paper is that it allowed me to give a response to most of the questions on this assignment, even if I was not able to find all the details in the paper. I admit that I do not find this topic particularly interesting, although the general process the author discusses is quite clever. 

\end{document}